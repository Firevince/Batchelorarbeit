\chapter{Anhang}\label{app:supplemental-information}


\label{graphql-1}
GraphQL Abfrage um alle download Links zu den Episoden des Podcasts "Radio Wissen" zu erhalten
Die Programset-ID verweist auf den speziellen Podcast.

\begin{verbatim}
{
    programSet(id: 5945518) {
        title
        items(
            orderBy: PUBLISH_DATE_DESC
            filter: {
            isPublished: {
                equalTo: true
            }
            }
        ) {
            
            nodes {
                title,
                audios {
                    downloadUrl
                }
            }
        }
    }
}

\end{verbatim}

\label{ch:graphql-2}
GraphQL Abfrage um alle Metadaten zu den Episoden des Podcasts "Radio Wissen" zu erhalten
Die Programset-ID verweist auf den speziellen Podcast.

\begin{verbatim}
    {
        programSet(id: 5945518) {
        title
        items(
            orderBy: PUBLISH_DATE_DESC
            filter: {
            isPublished: {
                equalTo: true
            }
            }
        ) {
          	
            nodes {
              title,
              keywords
              publishDate
              description
            }
        }
        }
    }

\end{verbatim}

\label{ch:graphql-3}
GraphQL Abfrage um alle transkripte zu den Episoden des Podcasts "Radio Wissen" zu erhalten
Die Programset-ID verweist auf den speziellen Podcast.

\begin{verbatim}

    {
    programSet(id: 5945518) {
        title
        items(
        orderBy: PUBLISH_DATE_DESC
        filter: {
            isPublished: {
            equalTo: true
            }
        }
        ) {
        nodes {
            transcript {
            data
            }
            audios {
            downloadUrl
            }
        }
        }
    }
    }
\end{verbatim}

\label{ch:chatgpt-reranking}

Ein Beispielprompt um Segmente von ChatGPT sortieren zu lassen.
Die einzelnen Segmente werden automatisch ergänzt.

Systemprompt:
\begin{itshape}
    Bringe die Abschnitte in eine logische Reihenfolge.
    Antworte in JSON Format als ein Array 'Reihenfolge' in welchem nur die Nummern der Segmente stehen.
\end{itshape}

Userprompt:

\begin{itshape}
    1: Und das waren ja hochgebildete und bedeutsame Personen unter ihnen, die da für das holländische Leben eine große Rolle spielten. Zentrum der neu gegründeten Republik war Amsterdam. Das mächtige, prächtige Amsterdam. Die reichste Stadt der Welt. Metropole eines weltumspannenden Handelsimperiums und so multikulturell wie vor ihm nur das alte Rom gewesen war. Alle Pracht und alle Luxusgüter aus aller Herren Länder kamen nach Amsterdam.
    
    2: Keine falsche Assoziation. Peter der Große hat in seiner Jugend die Niederlande besucht. Amsterdam gefiel ihm. Petersburg zu bauen, war aber keine nostalgische Entscheidung, sondern hatte machtstrategische Gründe. Dem Zaren lag Moskau als Hauptstadt zu weit im Osten. Der Nordriese, wie Peter manchmal genannt wird, wollte sein Reich europäischer machen, auch durch den Bau einer neuen Hauptstadt.
    
    3: Bis zu meinem vierten Lebensjahr wohnte ich in Frankfurt. Da wir Juden sind, ging dann mein Vater 1933 in die Niederlande. Amsterdam, das bedeutet unbeschwerte Kinderjahre. Bis die Nazis im Mai 1940 auch die Niederlande besetzen. Die Schlinge zieht sich zu für die Juden dort. Eine antisemitische Verordnung jagt die andere.
\end{itshape}

Beispielantwort:
(Die Antworten können variieren, da für die Inferenz kein Seed benutzt wurde)

\begin{verbatim}

    {
        "Reihenfolge": [2, 1, 3]
    }

\end{verbatim}



\label{ch:chatgpt-topics}

Ein Beispielprompt um weiterführende Themen von ChatGPT sortieren zu lassen.
Die einzelnen Segmente werden automatisch ergänzt.

Systemprompt:
\begin{itshape}
    Schreibe 5 weiterführende Themenüberschriften, die zu diesen Textabschnitten passen.
    Sie sollen spezifisch auf eine Sache aus diesem Abschnitt abzielen, oder ein ähnliches Thema beschreiben. 
    Jede Themenüberschrift soll aus 1-3 Wörtern Stichwörtern bestehen.
    Antworte in JSON Format als ein Array 'Themen' in welchem nur die Themen stehen 
\end{itshape}

Userprompt:

\begin{itshape}
    1: Keine falsche Assoziation. Peter der Große hat in seiner Jugend die Niederlande besucht. Amsterdam gefiel ihm. Petersburg zu bauen, war aber keine nostalgische Entscheidung, sondern hatte machtstrategische Gründe. Dem Zaren lag Moskau als Hauptstadt zu weit im Osten. Der Nordriese, wie Peter manchmal genannt wird, wollte sein Reich europäischer machen, auch durch den Bau einer neuen Hauptstadt.
    
    2: Und das waren ja hochgebildete und bedeutsame Personen unter ihnen, die da für das holländische Leben eine große Rolle spielten. Zentrum der neu gegründeten Republik war Amsterdam. Das mächtige, prächtige Amsterdam. Die reichste Stadt der Welt. Metropole eines weltumspannenden Handelsimperiums und so multikulturell wie vor ihm nur das alte Rom gewesen war. Alle Pracht und alle Luxusgüter aus aller Herren Länder kamen nach Amsterdam.
    
    3: Bis zu meinem vierten Lebensjahr wohnte ich in Frankfurt. Da wir Juden sind, ging dann mein Vater 1933 in die Niederlande. Amsterdam, das bedeutet unbeschwerte Kinderjahre. Bis die Nazis im Mai 1940 auch die Niederlande besetzen. Die Schlinge zieht sich zu für die Juden dort. Eine antisemitische Verordnung jagt die andere.
\end{itshape}

Beispielantwort:

\begin{verbatim}
    {
        "Themen": [
            "Machtstrategie Peters",
            "Amsterdams Reichtum",
            "Holländisches Handelsimperium",
            "Jüdisches Exil",
            "Nazibesatzung Niederlande"
        ]
    }
\end{verbatim}


\label{ch:chatgpt-boundaries}

Ein Beispielprompt um einen zum Thema passenden wichtigen Ausschnitt aus einem Podcast zu generieren.

Systemprompt:

\begin{verbatim}
    Schau dir das folgende Dokument an und wähle aus, welcher Abschnitt zur Beantwortung der Frage geeignet ist.

    {document}

    Antworte in JSON Format mit einem Text 'Begründung', der beschreibt warum dieser Textausschnitt dazu passt. 
    Außerdem einen 'Start' für die Nummer des Satzes, ab dem der Abschnitt beginnt und 
    einem 'Ende' für die Nummer des Satzes wo der Abschnitt aufhört.
    Wenn es keinen Abschnitt gibt der wirklich passt, antworte mit 'Start': 0 und 'Ende':0

\end{verbatim}

Userprompt:

\begin{verbatim}
  

\end{verbatim}


\label{ch:chatgpt-topicgeneration}

Ein Beispielprompt um Themen zur Evaluation der Embeddingmodelle zu erstellen.

\begin{verbatim}
Generiere 30 verschiedene Suchthemen, die aus den Themenbereichen Geschichte, Naturwissenschaft, Gesellschaft oder Philosophie kommen und auch über die deutsche Kultur handeln.
Die Themen sollen 1-3 verschiedene Wörter umfassen.
\end{verbatim}

\label{ch:chatgpt-questiongeneration}

Ein Beispielprompt um Fragen zur Evaluation der Embeddingmodelle zu erstellen.

\begin{verbatim}
Kannst du 30 Fragen generieren, wie "Wann werden Menschen auf dem Mars sein?" oder "Wo wird urbaner Gartenbau betrieben?"
\end{verbatim}

\label{ch:chatgpt-evaluation}

Ein Beispielprompt um Retrieval Ergebnisse zu evaluieren.

System-prompt:

\begin{verbatim}
    Du bewertest die Ausgabe eines Text Retrieval Systems. Bewerte, wie gut die gefundenen Textabschnitte zu dem gestellten Thema passen. Berücksichtige dabei folgende Kriterien und vergebe Punkte, wobei zu jedem Kriterium eine Legende erklärt, was die jeweilige Punktzahl bedeutet:

    - Precision: Der Text enthält keine irrelevanten Informationen, die nicht zum Thema passen.
      0 Punkte: Der Text ist völlig irrelevant.
      1-2 Punkte: Der Text enthält größtenteils irrelevante Informationen.
      3-4 Punkte: Der Text ist überwiegend relevant mit einigen irrelevanten Details.
      5 Punkte: Der Text ist vollständig relevant ohne irrelevante Informationen.
    
    - Cohesion: Die einzelnen Textausschnitte passen inhaltlich zusammen.
      0 Punkte: Kein Zusammenhang zwischen den Textausschnitten.
      1 Punkt: Geringer Zusammenhang zwischen den Textausschnitten.
      2 Punkte: Hoher Zusammenhang und gute inhaltliche Verknüpfung zwischen den Textausschnitten.
    
    - Uniqueness: Die Textausschnitte wiederholen keine Informationen.
      0 Punkte: Starke Wiederholungen von Informationen.
      1-2 Punkte: Einige Wiederholungen, aber auch neue Informationen.
      3 Punkte: Jeder Textausschnitt liefert einzigartige und neue Informationen.
    
    Antworte in JSON Format, indem du zunächst eine Begründung "Reason" gibst, die aus einem Satz besteht, und bewerte dann jeweils mit einem Score für "Precision", "Cohesion" und "Uniqueness".    
\end{verbatim}

User-Prompt:

\begin{verbatim}  
    Textabschnitte: {documents}

    -----------

    Frage: {query}
\end{verbatim}

