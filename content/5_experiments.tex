\chapter{Evaluation verschiedener semantischer Verfahren}\label{ch:experiments}


\section{BERT Embeddings}

Um einen Embeddingvektor zu erstellen, benutzen wir das BERT Model. BERT, das Akronym für Bidirectional Encoder Representations from Transformer, ist ein Sprachmodel, das 2018 von Google entwickelt, und zur Benutzung freigegeben wurde. 
BERT ist ein Neuronales Netzwerk mit 12 Schichten, das für zwei verschiedene Aufgaben gleichzeitig trainiert wurde. Zum einen wurde es auf eine Masked Language Modeling Aufgabe und zum Anderen auf eine Next Sentence Prediction trainiert. 

Masked Language Modeling
Bei dieser Aufgabe soll das Model versuchen, aus dem Kontext eines Satzes ein maskiertes Wort in diesem Satz vorherzusagen. Dafür wird dem Model ein Satz gegeben, in dem zufällige Wörter einfach versteckt werden und das Model soll für diese Wörtereine Vorhersage treffen. 


\section{Sentence Transformer Embeddings}

Das Sentence Transformer Projekt baut auf der Architektur von BERT auf. 
Es wird auch SBERT für Sentence BERT genannt. 
Unter dem 

\section{LLama2 Embeddings}

Ein sehr Leistungsfähiges LLM ist das von Meta entwickelte LLama und deren Nachfolger LLama2. \cite{touvron2023}

\section{OpenAI Embeddings}

\href{https://twitter.com/Nils_Reimers/status/1487014195568775173}{twitter}



OpenAI stellt außer den Text (GPT-4) und Bild (DALLE) generierungs Modellen auch Modelle zur Embedding erstellung in Ihrer API vor.
Ada (1024 dimensions)
Babbage (2048 dimensions)
Curie (4096 dimensions)
Davinci (12288 dimensions)

\section{Auswertung}


Die Aufgabe besteht darin, zu untersuchen, wie gut eine Suchfunktion mithilfe der Embeddings in der Lage ist, passende Informationen zu einem Thema zu extrahieren.

Für die Evaluation der verschiedenen Embeddings werden verschiedene Kriterien angelegt.
Einerseits müssen die extrahierten Segmente zu dem Thema passen und nicht irrelevante andere Themen behandeln.
Das ist der wichtigste Punkte bei der Evaluation.

Ein weiteres Kriterium ist der Inhaltliche Zusammenhang der einzelnen Segmente.
Die einzelnen Segmente sollten eine inhaltliche Verbindung aufweisen, sodass der/die Hörer*in nicht zwischen verschiedenen Themn hin und hergeworfen wird.
Zum Beispiel soll ein Podcast über "Geschichte Amsterdam" nicht in einem Segment über die Gründung der Niederlande sprechen und im nächsten Segment davon handeln, wie sich Amsterdam in der Nazi Zeit verhalten hat.
Gleichzeitig sollen die Segmente auch nicht alle den selben Inhalt aufweisen und zum Beispiel in drei verschiedenen Segmenten wiederholen, dass die Niederlande im goldenen Zeitalter des 17. Jahrhunderts ein florierendes Handelsnetzwerk aufgebaut hatten.
 
Die Themen, nach welchen die Embeddings evaluiert werden sollen sind in drei Kategorien eingeteilt:
topical, refinding, known items \cite{jones2021}
