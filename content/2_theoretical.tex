\chapter{Ausgangslage und theoretische Grundlagen}\label{ch:data}

\section{Ausgangslage}
 
Die Podcastbranche wächst seit vielen Jahren stetig und immer mehr Menschen hören regelmäßig Podcasts 
In Deutschland ist die ARD Audiothek ein großer Podcastanbieter mit mitlerweile über 41 Millionen Audioabrufen und über 80.000 verschiedenen Audioinhalten zum Abrufen. Auch die Zahlen der Audiotheksbenutzer, sowie der App-downloads steigen weiterhin. \cite{gotting2023}

Gleichzeitig wächst auch der Markt an AI-basierten Podcasts. So erreichen heute schon AI-basierte Podcasts rund 45 Millionen US-Amerikaner 

\section{KI in Podcasts}

Künstliche Intelligenz verändert die Branche des Podcastings in vielerlei Hinsicht. 

Transkriptionsmodelle wie Whisper sind in der Lage live Transkriptionen der Podcasts zu erstellen die eine Qualität bestitzen, die mit professionellen menschlichen Transkriptoren mithalten kann \cite{radford}.
Diese können außerdem verschiedene Stimmen unterscheiden und die Emotionen der Sprecher erkennen, was es in der Nachbereitung eines Podcasts erheblich erleichtert bestimmte Stellen zu finden \cite{wagner2023}
Ein großes Entwicklungsfeld in der Podcastbranche ist die komplett automatische Generierung von Podcasts. Die Technologie der automatischen Stimmengenerierung ist soweit fortgeschritten, dass sich künstlich generierte Stimmen fast so gut anhören wie echte Stimmen \cite{shi2023}.


\section{Embeddings}

Das Ziel dieser Arbeit besteht darin, eine Schnittstelle für einen Hörer zu erstellen, die ihm ermöglicht, einen Zusammenschnitt aus vielen verschiedenen Podcast Episoden ist, und genau zu einem Thema passt. 
Die wichtigste Aufgabe besteht also darin, die wesentlichen Abschnitte aus den Episoden herauszufinden. 
Bis jetzt haben wir zu jedem Audiofile eine transcript.json, bei der das Transkript in regelmäßigen Zeitabschnitten mit Zeitstempeln markiert ist. 
Die Aufgabe, auf eine Userfrage hin die wesentlichsten Segmente herauszufinden lässt sich auf mehrere Arten lösen. 
Dazu werden wir verschiedene Verfahren aus der Wissenschaft im Bereich Natural Language Processing (NLP) verwenden.

\section{Motivation für Embeddings}

Die menschliche Sprache ist ein hochkomplexes Konstrukt mit einer Grammatik, die sehr viel flexibler, kreativer, vieldeutiger, und komplexer ist als maschinensprache. 
Es gibt viele kleine Bedeutungsnuancen, sie ist stark von dem allgemeinen Wissen der Welt geprägt und sie ändert sich im laufe der Zeit. 
Das alles macht es für Computer sehr schwierig die Menschliche Sprache zu Verstehen. 
Neuere Forschung im Bereich des Natural Language Processing bietet einige Ansätze um dieses Problem zu lösen. 
Es gibt dazu viele verschiedene Methoden, die alle darauf abzielen, Worte oder Texte in Vektoren umzuwandeln, die den Inhalt dieser Worte oder dieser Texte repräsentiert.
Diese Vektoren nennt man Embeddings.
Das Ziel dieser Embeddings besteht für unseren Zweck darin, eine Suchfunktion zu beschreiben, die auf eine Anfrage (Query) hin, passende Dokumente aus einem großen Korpus an Dokumenten liefern kann.

Die Geschichte der Embeddings startet mit stochastischen Verfahren, wie TF-IDF und BM25. 
Term Frequency - inverse Dokument Frequency (TF-IDF) ist ein Algorithmus um Dokumente nach Keyworten zu durchsuchen.
Das Verfahren verwendet Worthäufigkeiten (Term-Frequency), um Dokumenten, in denen ein Wort häufiger vorkommt zu bevorzugen.
Außerdem verwendet es die generelle Seltenheit von Wörter (Inverse Document Frequency) um Wörtern, die seltener vorkommen mehr Gewicht bei der Suche einräumen.

Daraus bildet der TF-IDF Algorithmus einen dünnbesetzten Vektor, der die Anzahl und Seltenheit jedes im Dokument vorkommendes Wortes abbildet. 
Dünnbesetzt heißt in diesem Zusammenhang, dass der Vektor viele Einträge mit dem Wert 0 besitzt. 
Das folgt aus der Bedingung, dass diese Vektoren untereinander vergleichbar sein müssen und dadurch Einträge für jedes Wort besitzen, welches im gesamten Korpus an Dokumenten vorkommt.
Jedes einzelne Dokument hat dabei aber nur ein Bruchteil der Gesamtwörter und damit viele Nullwerte.

Best Matching 25 (BM25) ist eine Famile von Algorithmen die versuchen den TF-IDF Algorithmus zu verbessern indem sie versuchen, die Priorisierung Langer Dokumente und zu oft vorkommende Worte auszugleichen.

Außerdem kann man diese Algorithmen noch anpassen, sodass sie auch n-gramme als Einträge der Vektoren miteinbeziehen. 
Das heißt, dass nicht nur einzelne Wörter berücksichtigt werden, sondern auch häufig zusammen vorkommende Wörter als einzelnes Token in dem Vektor abgebildet werden. 
Das bewirkt zwar, dass die Ähnlichkeit der Dokumente in der Regel besser ermittelt werden kann, vergrößert aber die Vektoren überproportional.

Semantische Analysen begannen erst ab ca. 1990 mit der Latenten Semantischen Analyse (LSA).
Diese verwendet Eigenvektoren um aus einer Term-Frequency Matrix versteckte (latente) Eigenschaften zu ermitteln, welche die Dokumente besser repräsentieren als die TF Matrix an sich. 
Es werden von diesen Eigenvektoren nur die wichtigsten ausgewählt, sodass jedes Dokument sehr gut als eine linearkombination dieser Vektoren repräsentiert werden kann.
Dann werden nur noch die Koeffizienten dieser Vektoren gespeichert und man kann von einer Dimensionsreduktion profitieren.
Die resultierenden Koeffizienten-Vektoren sind nun nicht mehr spärlich besetzt sondern Dicht.
Die resultierenden Eigenvektoren bilden häufig latente Themen der Dokumente ab.

Wirkliche Word Embeddings kamen erst mit der Veröffentlichung von Word2Vec (2013) in den breiteren Nutzen. 
Word2Vec verwendet ein neuronales Netz, um mithilfe der benachbarten Wörter Kontextinformationen über das eigentliche Wort zu erhalten. 
Dafür gibt es die beiden Ansätze continous Bag of Words und Skipgramm.
Bei dem Ansatz Continous Bag of words werden dem Neuronalen Netz die umgebenden Wörter als Input gegeben, und das Model hat die Aufgabe daraus das Wort zu ermitteln. 
Dieses Verfahren wird in einem Sliding Window für jedes Wort aus einem Text wiederholt. 
Für einen bestimmten Korpus aus verschiedenen Dokumenten kann dieses Modell dadurch die beziehungen verschiedener Worte zueinander und die Ähnlichkeit verschieddener Worte, die oft in gleichem Kontext vorkommen erlerenen. 
Das Verfahren des Skipgramms funktioniert ähnlich, allerdings wird dabei für jedes Wort versucht die umliegenden Wörter zu ermitteln. 
Dieser Ansatz dauert länger im Training, hat aber den Vorteil, dass er auch seltener vorkommende Wörter gut repräsentieren kann.

Ein Jahr später wurde GloVe (Global Vectors for Word Representations) als Forschungsprojekt von der Stanford Universität vorgestellt. 
GloVe verbindet Konzepte aus LSA und Word2Vec, indem es die auf einer globalen Matrix Faktorisierung beruht aber die Distanz von Wörtern zueinander mitberücksichtigt.
Dadurch schaffte GloVe eine weitere Verbesserung der Embeddings und wird heutzutage noch unter anderem in NLP Bibliotheken wie SpaCy verwendet.[englische Wikipedia referiert auf Buch aus 2018]

Im Jahre 2018 wurde der Ansatz ElMo (Embeddings from Language Model) für Word Embeddings vorgestellt.
Dieser benutzt ein Bidirectional Long Short Term Memory Model (LSTM) um daraus Embeddings zu generieren.
Long Short Term Memory Modelle basieren auf Recurenten Neuronalen Netzen (RNN) und gehören zu den Sequenz to Sequenz Modellen. 
Sie bestehen dabei aus einem Encoder und einem Decoder.
Der Encoder codiert eine Inputsequenz, zum Beispiel ein Dokument, zu einem Embeddingvektor. 
Der Decoder dekodiert diesen Embeddingvektor wieder zu einer Sequenz, z.B. einer zusamenfassung des Dokuments.
Für ELMo ist dabei nur der Encoder wichtig.










Der Wunsch die menschliche Sprache für den Computer verständlich zu machen ist fast so alt, wie die Computer an sich. 
In den 1940er Jahren, nach Ende des 2. Weltkrieges 
Der Bereich des NLP riesig. 
Es gibt [QUELLE] paper dazu. 
Es hat viele verschiedene Unterthemen. 
Die Aufgaben im Bereich des NlP sind vielseitig . Einige Ansätze 

\section{Ähnlichkeitsvergleiche}
\section{Transformer Architektur}


Eine Transformerarchitektur ist eine der modersten und leistungsfähigsten Architekturen, um NLP Aufgaben zu lösen. 
Sie bildet dabei den Nachfolger bzw. Konkurenten zu den bis dato vorherschenden Rekurenten Neuronalen Netzen (RNN) oder Long-Short Term Memory Systemen (LSTM). 
Der Große Vorteil der Transformer ist, dass sie parallelisierbarer sind, da RNNs häufig mehrschrittige rekurente Abfragen bilden müssen.
Das heißt pro Token Output müssen Sie unter Umständen mehrfach das Embedding des Inputs durchsuchen. Audio Segmentierung Die Audios müssen am Ende noch zugeschnitten werden. 



\section{Large Language Model}

\section{Diffusion Model}