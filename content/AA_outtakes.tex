\section{Erstellung einer Podcast Episode}

Erfolgreiche Podcasts zu erstellen benötigt viele einzelne Schritte.
Thema:
Am Anfang benötigt man immer eine Idee, die den Podcast bestimmt.
Das kann die Abhandlung eines Themas oder das halten an ein bestimmtes Format sein.
Vorbereitung:
Dann muss eventuell ein Skript vorgeschrieben werden, Gäste eingeladen werden oder das die Audio-Aufnahmebedingungen geklärt werden.
Zur vorbereitung von INterview Podcasts muss sich auf den jeweiligen gegenüber vorbereitet, Fragen formuliert und hintergrundwissen recherchiert werden.
Aufnahme:
Für die Aufnahme müssen die Sprecher vor Ort oder Remote sprechen.
Es gibt Audio und Video Podcasts
Nachbereitung/Editierung:
Zur Nachbereitung gehört das Anpassen der Lautstärke, herausschneiden von ungewünschten Abschnitten, Einfügen von Geräuschen oder zusätzlichen passagen.
Shownotes erstellen:
Der Podcast sollte transkribiert werden und eine Zusammenfassung erstellt werden.
Showbild erstellen:
Ein Bild für diese Episode erstellen.
Werbung schalten:
Kleine Teaser erstellen, die man z.B. auf Social Media teilen kann.


\section{KI in Podcasts}

Künstliche Intelligenz verändert die Branche des Podcastings in vielerlei Hinsicht. 

Transkriptionsmodelle wie Whisper sind in der Lage live Transkriptionen der Podcasts zu erstellen, die eine Qualität besitzen, die mit professionellen menschlichen Transkriptoren mithalten können \cite{radford}.
Diese können außerdem verschiedene Stimmen unterscheiden und die Emotionen der Sprecher erkennen, was es in der Nachbereitung eines Podcasts erheblich erleichtert, bestimmte Stellen zu finden \cite{wagner2023}
Es gibt Ansätze, KI gestützt Teaser von längeren Podcast Episoden zu erstellen, welche dann zu Werbezwecken in Sozial Media geteilt werden können. \cite{wang2023}

Ein großes Entwicklungsfeld in der Podcastbranche ist die komplett automatische Generierung von Podcasts. Die Technologie der automatischen Stimmengenerierung ist soweit fortgeschritten, dass sich künstlich generierte Stimmen fast so gut anhören wie echte Stimmen \cite{shi2023}.

Laut der Plattform Podcastle, die Software für Aufnahme und Editierung von Podcasts herstellt, 
erreichen heute schon AI-basierte Podcasts rund 45 Millionen US-Amerikaner \cite{podcastle2023}.
Der Podcast "Hacker News Recap" ist ein vollkommen von AI produzierter Podcast und erreicht in vielen Ländern wie Schweden oder Italien die Top 100 bei Apple Podcasts in der Kategorie Daily News \cite{chartable}.
In Deutschland ist der Podcast immerhin auf Platz 169.