\chapter{Einleitung}\label{ch:intro}

\section{Motivation}

Podcasts sind für viele Menschen ein wichtiges Medium, um sich die Zeit zu vertreiben und sich über verschiedene Themen zu informieren. 
In Deutschland hören ca. 29\% der Menschen regelmäßig Podcasts \cite{newman2022}.
Dabei liegt die Nutzung der ARD-Audiothek mit 12\% auf Platz drei hinter Spotify und Youtube[TODO Quelle].
Wie Studien gezeigt haben, verbessert sich das Audioerlebnis der Zuhörer*innen dabei, wenn mehrere verschiedene Sprecher*innen dabei vorkommen \cite{kang2012}. 
Einige Podcasts werden zum Beispiel nur von einer einzigen Person vorgetragen.
Außerdem steigert das Vorhandensein verschiedener Meinungen auch das Interesse des Zuhörers an einem bestimmten Thema \cite{phillips2014}.
Neben diesen Vorteilen, ist es für die Zuhörer*innen einer Podcast Episode auch ein besonderer Aspekt der Selbstbestimmung, wenn sie die Länge einer Podcast Episode selbst festlegen können.
In dieser Arbeit wird es darum gehen, wie man diese Vorteile einer personalisierten Podcast Episode ausnutzen kann.


\section{Zielsetzung}


Das Ziel dieser Bachelorarbeit besteht darin zu untersuchen, wie sich aus umfangreichem Audiomaterial aus Radioprogrammen oder Podcasts on-the-fly ein eigener Podcast zusammenstellen lässt, der relevante Ausschnitte aus einer Vielzahl von Audiomaterial enthält.

Ein möglicher Anwendungsfall wäre ein/e Benutzer*in, die/der sich über das Thema "Überfischung der Meere" informieren möchte und dafür genau 20 Minuten während einer Autofahrt einplant. 
Das System erstellt nun einen Zusammenschnitt aus verschiedenen Podcast Episoden zu diesem Thema, der 20 Minuten lang ist und stellt ihn dem/der Benutzer*in zur Verfügung. 
Der Vorteil für den/die Nutzer*in liegt darin, dass er/sie selbst das Thema auswählen und die exakte Länge festlegen kann, um beispielsweise während einer 20-minütigen Autofahrt einen Podcast anzuhören. 
Außerdem werden das Thema von verschiednen Personen aus unterschiedlichen Blcikpunkten erklärt. 

Für die Interaktion mit dem Benutzer soll außerdem eine Grafische Benutzeroberfläche bereitgestellt werden, die dem Nutzer die Auswahl eines Themas und die Länge der Podcast  Episode ermöglicht.

\section{Überblick}

Im zweiten Kapitel werden theoretische Grundlagen zum Verständnis der später aufgeführten Technologien erklärt.

Im dritten Kapitel wird es um die Beschaffung der Datengrundlage gehen. Als Datenquelle werden Podcast Episoden aus der Audiothek der ARD benutzt. 
Dann werden verschiedene Methoden zur Transkription dieser Audiodaten beschrieben und begründet, welche Wahl der Transkription für diese Arbeit verwendet wird.

Im vierten Kapitel wird die Architektur des Systems dargestellt. 
Dabei wird besonders auf die semantische analyse der Transkriptionen eingegangen und verschiedene Aspekte des Natural Language Processing und der Large Language Models vorgestellt.

Im fünften Kapitel werden verschiedene Methoden zur semantischen Analyse evaluiert.
Dabei werden auch die verwendeten Modelle die Wahl der Anzahl und Länge der Abschnitte und die finale Zusammensetzung dieser Abschnitte erklärungt und begründet.

Im sechsten Kapitel wird ein Ausblick auf zukünftige Verbesserungen gegeben.

Im siebten Kapitel wird eine Zusammenfassung der erzielten Ergebnisse gegeben.
