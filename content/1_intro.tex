\chapter{Einleitung}\label{ch:intro}

\section{Motivation}

\section{Zielsetzung}


Das Ziel dieser Bachelorarbeit besteht darin zu untersuchen, wie sich aus umfangreichem Audiomaterial aus Radioprogrammen oder Podcasts on-the-fly ein eigener Podcast zusammenstellen lässt, der relevante Ausschnitte aus einer Vielzahl von Audiomaterial enthält.

Ein möglicher Anwendungsfall wäre ein Benutzer, der sich über das Thema "Überfischung der Meere" informieren möchte und dafür genau 20 Minuten während einer Autofahrt einplant. Das System erstellt nun einen zusammenschnitt aus verschiedenen Podcasts zu diesem Thema der 20 Minuten lang ist und stellt ihn dem Benutzer zur Verfügung. Der Vorteil für den Nutzer liegt darin, dass er selbst das Thema auswählen und die exakte Länge festlegen kann, um beispielsweise während einer 20-minütigen Autofahrt einen Podcast anzuhören.

Für die Interaktion mit dem Benutzer soll außerdem eine Grafische Benutzeroberfläche bereitgestellt werden, die dem Nutzer die Auswahl eines Themas und die Länge des Podcasts ermöglicht.

\section{Überblick}





