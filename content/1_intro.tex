\chapter{Priorisierung der Literatur}\label{ch:intro}

Die wichtigste Literatur stellt der Artikel von \cite{laban2022} dar, da er erst vor einem Jahr erschienen ist und sich auch mit der automatischen Generierung von Podcasts befasst. \cite{laban2022} beschäftigt sich mit der Extrahierung von Semantischen Informationen aus Nachrichtenberichten und der Aufbereitung dieser in Form von sprachsynthetisierten Podcast Episoden.  Dadurch wird dieser Artikel sehr relevant für meine eigene Bachelorarbeit werden. 

Als zweites würde ich die Arbeit von \cite{zhang2020} priorisieren, da sie als Vorarbeit von \cite{laban2022} dient und das wichtige Summarization-Modell PEGASUS vorstellt, das ich auch in meiner Arbeit verwenden kann.

Als weitere Quellen werde ich \cite{karpukhin2020}, \cite{reddy2019} und \cite{choi2018} verwenden, die sich alle mit der automatischen Fragen- und Antwortgenerierung aus Texten beschäftigen.
\cite{karpukhin2020} ist dabei der neueste Ansatz zur Ermittlung von Kontext, der sich außerdem von dem TF-IDF Algorithmus unterscheidet und sehr gute Ergebnisse verspricht.

Für das Design der Podcast Episoden werde ich die Artikel von Podcast Übersicht von \cite{jones2021} verwenden, in dem mithilfe von Deep Learning Ansätzen mehr als 100.000 Podcast Episoden analysiert wurden und daraus Aspekte für eine erfolgreiche Podcast Episode kristalisiert wurden. Diese Aspekte werden mir bei der Gestaltung und dem Aufbau einer Episode helfen. 

In dem Paper \cite{kang2012} untersuchten die Autoren den Effekt von mehrere Stimmen auf das Podcasterlebnis. Außerdem untersuchten Sie in Ihrer Studie die Effekte des Erzählstils und finden heraus, dass informelle Sprache und ein ungezwungener Kommunikationsstil das Erlebnis von Zuhörern verbessert. 

Im weiteren werde ich noch die Arbeiten von \cite{maroni2020}, \cite{clark2020} und \cite{du2017} verwenden, die sich mit Deep Learning Modellen zur Erschließung von Texten beschäftigen.


