\thispagestyle{empty}
\chapter{Textzusammenfassung}\label{ch:summary}

In dem Artikel \glqq NewsPod: Automatic and Interactive News Podcasts\grqq{} \cite{laban2022} wird ein neuer Ansatz für die automatische Erstellung von Podcast Episoden vorgestellt. 
Dabei ermitteln mehrere Machine Learning Systeme den Inhalt von bestimmten Nachrichtenseiten und extrahieren daraus Fragestellungen, die den Inhalt des Textes wiederspiegeln sollen. Die Autoren benutzen ein GPT-2 Sprachmodell, das auf Fragengenerierung trainiert wurde, um aus jedem Absatz 7 Fragen zu ermitteln. 
Für jede Frage wird dann eine Seperate Antwort generiert. 
Der Wichtigste Aspekt an diesem Artikel ist, dass die Benutzer der Software Interaktiv agieren können und selbstgewählte Fragen mithilfe eines Mikrofons oder einer Tastatur stellen können, die dann von dem System beantwortet werden. 
Die Autoren dieses Papers führen außerdem zwei Studien zur Nutzung dieses Systems durch.
Der Gegenstand der ersten Studie ist das die Zufriedenheit einer Testgruppe mit der Erzählweise des Textes und der automatisch generierten Stimme des Sprechers. 
Die zweite Studie untersucht die Interaktion der Zuhörer während der Benutzung des Systems. 
\cite{laban2022}

In der ersten Studie konnten die Autoren feststellen, dass einer ihrer Ansätze, QA Best,  so erfolgreich war, dass \glqq  80\% of QA Best listeners said they would use the system to listen to the news in the future\grqq{}  \cite{laban2022}.
Die zweite Studie kommt zu dem Schluss, dass zwar die Bereitschaft eigene Fragen zu stellen mit 85\% der Zuhörer sehr hoch war, die Zufriedenheit der Tester mit der Qualität der Antworten aber sehr gering ausfiel, da \glqq  76\% of the answers were rated Irrelevant/Confusing\grqq{} \cite{laban2022}.




