\chapter{Ausblick}\label{ch:outlook}

Dieses Kapitel bietet einen Ausblick auf mögliche Verbesserungen des vorgestellten Podcast-Generierungssystems und skizziert zukünftige Entwicklungspfade.

Für eine weitergehende Optimierung des Systems könnten zunächst die zugrunde liegenden Datenbestände erweitert werden. Mehr Daten würden die Vielfalt und Tiefe des Inhalts erhöhen, was zu einer qualitativ besseren und nuancierteren Podcast-Produktion führen kann. Des Weiteren wäre es sinnvoll, die Segmentierung der Daten zu überarbeiten. Größere und überlappende Segmente könnten dabei helfen, den Kontext besser zu bewahren und die Verknüpfung von relevanten Inhalten zu erleichtern.

Ein weiteres wichtiges Feld für zukünftige Entwicklungen ist die Exploration neuer Embeddingmodelle. Durch das Testen und Integrieren verschiedener fortschrittlicher Embeddingmethoden könnte die Effizienz und Präzision des Retrievalprozesses deutlich verbessert werden.

Die Zukunft des Information Retrievals könnte sich jedoch grundlegend ändern, wie die Forschung im Bereich "Mistral of Experts" \cite{jiang2024} nahelegt. Hier wird darauf hingewiesen, dass durch den Einsatz innovativer Technologien, die große Kontextmengen verarbeiten können, die traditionellen Methoden des Information Retrievals möglicherweise bald überholt sein könnten. Insbesondere die neuen Modelle wie Gemini1.5 Pro \cite{2024d}, die über die Kapazität verfügen, mehr als 10 Millionen Token zu verarbeiten, zeichnen sich durch eine außergewöhnlich hohe Retrieval-Rate aus. Diese Entwicklungen deuten darauf hin, dass wir an der Schwelle zu einer neuen Ära im Bereich des Information Retrievals stehen, in der die Grenzen der aktuellen Systeme signifikant erweitert werden könnten.





