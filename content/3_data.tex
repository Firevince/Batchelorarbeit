\chapter{Datenbeschaffung}\label{ch:data}

\section{Die ARD-Audiothek}

Die ARD-Audiothek ist eine AUdiodatenbank der ARD, die öffentlich rechtlich ist und Ihre Inhalte zur Verfügung stellt.

\section{Die ARD-Audiothek API}

Die Inhalte in der ARD Audiothek kann man entweder dirkt über die die Webseite erreichen, oder mithilfe einer frei benutzbare  GraphQL API abfragen. über diese Schnittstelle bekommt man alle Informationen, wie den Titel einer Episode, die Beschreibungen, die Autoren und auch den Link zu dem mp3 file.
Diese Schnittstelle wird verwendet, um später die Audiodatein zu erhalten. 
Über die API kann auch in einigen Fällen direkt ein Transkript des Audiofiles angefordert werden. 
Allerdings ist die Transkription meist nicht sehr akkurat.
Zum Besipiel steht in der Transkription des Satzes ... TODO
Das liegt daran, das die Transkriptionen mithilfe der Tools vom Fraunhofer Instituts erstellt werden, welches vermutlich veraltete Technik benutzt.


